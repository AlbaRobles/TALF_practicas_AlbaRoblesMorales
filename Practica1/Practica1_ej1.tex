\documentclass[fleqn, 10pt]{article}

% Paquetes necesarios
\usepackage[utf8]{inputenc}
\usepackage[spanish]{babel}
\usepackage{amsthm}
\usepackage{nccmath} %Para centrar ecuaciones
\usepackage{enumitem}

% Personalizo mi alfabeto
\DeclareMathAlphabet{\pazocal}{OMS}{zplm}{m}{n}
\newcommand{\Lb}{\pazocal{L}}


\theoremstyle{plain}
\newtheorem{proposicion}{Proposición}

\theoremstyle{definition}
\newtheorem{definition}{Definición}[section]
\newtheorem{example}{Ejemplo}[section]

%Definimos el título
\title{Teoría de Autómatas y Lenguajes Formales\\[.4\baselineskip]Práctica 1: Latex y expresiones regurales}
\author{Alba Robles Morales}
\date{\today}



%Comienzo del documento
\begin{document}

%Generamos el título
\maketitle

\section{Encuentra el conjunto $R^3$ de $R$ =\{(1,1),(1,2),(2,3),(3,4)\} }

En primer lugar, debemos encontar $R^2$ :\\
\begin{center} 
$R^2$ = \{(1,1),(1,2),(1,3),(2,4)\}
\end{center}

Una vez tenemos $R^2$ podemos encontrar $R^3$ de la misma manera:\\
\begin{center} 
$R^3$= \{(1,1),(1,2),(1,3),(1,4)\}
\end {center}
\end{document}